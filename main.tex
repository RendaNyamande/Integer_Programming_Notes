\documentclass[12pt, a4paper]{article}
\usepackage[utf8]{inputenc}
\usepackage[]{graphicx}
\usepackage[]{amsmath}
\graphicspath{{figures/}}
\title{Integer Programming}
\author{Renda Nyamande}
\date{30 August 2022}

\begin{document}
\maketitle

\begin{abstract}
    \begin{center}
        \textbf{We shall conquer at all costs!}
    \end{center}
\end{abstract}
\pagebreak

\section*{What is it?}
\paragraph*{Def}
LP problems with at least one variable that has to be an integer.
\subsection*{Types}
\begin{itemize}
    \item Pure - All decision vars have to be integers.
        \begin{math}
            X\in int
        \end{math}
    \item Mixed - Some ints and some not (Even int + binary).
    \item Binary - Decision vars can only take on the values 0 or 1.\\
        \begin{math}
            X\in\{0;1\}
        \end{math}
\end{itemize}
\subsection*{Characteristics}
\begin{itemize}
    \item The change from a normal LP to an integer one changes the problem drastically; Making it more complex (Might not even have a solution).
\end{itemize}

\subsection*{Example 1}
\begin{figure}[ht]
    \centering
    \includegraphics*[width=0.75\textwidth]{example1}
    \caption{Example 1}
    \label{fig: Example 1}
\end{figure}
\pagebreak

Max $
    X_1 +3X_2\\
    S.t:\\
    2X_1+8X_2 \leq 9\\
    4X_1+X_2 \leq 15.5\\
    8X_1-4X_2 \geq -5\\
    X_1, X_2 \geq 0
$ and Integer\\
\begin{itemize}
    \item Point A would be the optimal solution point if this was a normal LP problem.
    \item Trying to use the same point rounded up would result in one using a point that's actually not in the feasible region.
    \item If you round down, the solution would be feasible but not optimal.
    \item Adding an int constraint can only restrict the solution.
    \item Therefore A is the best solution we could ever hope to achieve having restricted things.
    \item So shifting the isocost line down from A (Restricting solution), the first int point it touches will be the optimal point.
    \item Theres an algorithm we'll use to find these solutions.
\end{itemize}
\subsection*{Example 2}
\begin{figure}[ht]
    \centering
    \includegraphics*[width=1\textwidth]{example2}
    \caption{Example 2}
    \label{fig: Example 2}
\end{figure}

\begin{equation}
   Let X_j =
  \begin{cases}
    1 & \text{if project $j$ is selected; j = 1,2,3,4} \\
    0 & \text{Otherwise}
  \end{cases}
\end{equation}
Max $8000X_1+11000X_2+6000X_3+4000X4$
St:
$5000X_1+7000X_2+4000X_3+3000X_4 \leq 14000\\
X_1, X_2, X_3, X_4 \in \{0,1\} 
$
\begin{itemize}
    \item Using simplex, Z = R22000 ($X_1=X_2=1, X_3=0.5, X_4=0$)
    \item This obviously doesn't satisfy the last constraint.
    \item If you round down, Z = R19000
    \item Better solution - $X_1=0, X_2=X_3=X_4=1$
    \item This is found using the algorithm we'll learn later.
\end{itemize}

\end{document}