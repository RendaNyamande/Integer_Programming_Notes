\documentclass[12pt, a4paper]{article}
\usepackage[utf8]{inputenc}
\usepackage[]{graphicx}
\usepackage[]{amsmath}
\graphicspath{{figures/}}
\title{Integer Programming}
\author{Renda Nyamande}
\date{30 August 2022}

\begin{document}
\maketitle

\begin{abstract}
    \begin{center}
        \textbf{We shall conquer at all costs!}
    \end{center}
\end{abstract}
\pagebreak

\section*{What is it?}
\paragraph*{Def}
LP problems with at least one variable that has to be an integer.
\subsection*{Types}
\begin{itemize}
    \item Pure - All decision vars have to be integers.
        \begin{math}
            X\in int
        \end{math}
    \item Mixed - Some ints and some not (Even int + binary).
    \item Binary - Decision vars can only take on the values 0 or 1.\\
        \begin{math}
            X\in\{0;1\}
        \end{math}
\end{itemize}
\subsection*{Characteristics}
\begin{itemize}
    \item The change from a normal LP to an integer one changes the problem drastically; Making it more complex (Might not even have a solution).
\end{itemize}

\subsection*{Example 1}
\begin{figure}[ht]
    \centering
    \includegraphics*[width=0.75\textwidth]{example1}
    \caption{Example 1}
    \label{fig: Example 1}
\end{figure}
\pagebreak

Max $
    X_1 +3X_2\\
    S.t:\\
    2X_1+8X_2 \leq 9\\
    4X_1+X_2 \leq 15.5\\
    8X_1-4X_2 \geq -5\\
    X_1, X_2 \geq 0
$ and Integer\\
\begin{itemize}
    \item Point A would be the optimal solution point if this was a normal LP problem.
    \item Trying to use the same point rounded up would result in one using a point that's actually not in the feasible region.
    \item If you round down, the solution would be feasible but not optimal.
    \item Adding an int constraint can only restrict the solution.
    \item Therefore A is the best solution we could ever hope to achieve having restricted things.
    \item So shifting the isocost line down from A (Restricting solution), the first int point it touches will be the optimal point.
    \item Theres an algorithm we'll use to find these solutions.
\end{itemize}
\subsection*{Example 2 - Project selection problem}
\begin{figure}[ht]
    \centering
    \includegraphics*[width=1\textwidth]{example2}
    \caption{Project selection problem}
    \label{fig: Project selection problem}
\end{figure}

\begin{equation}
   Let X_j =
  \begin{cases}
    1 & \text{if project $j$ is selected; j = 1,2,3,4} \\
    0 & \text{Otherwise}
  \end{cases}
\end{equation}
Max $8000X_1+11000X_2+6000X_3+4000X4$\\
St:\\
$5000X_1+7000X_2+4000X_3+3000X_4 \leq 14000\\
X_1, X_2, X_3, X_4 \in \{0,1\} 
$
\begin{itemize}
    \item Using simplex, Z = R22000 ($X_1=X_2=1, X_3=0.5, X_4=0$)
    \item This obviously doesn't satisfy the last constraint.
    \item If you round down, Z = R19000
    \item Better solution - $X_1=0, X_2=X_3=X_4=1$
    \item This is found using the algorithm we'll learn later.
    \item We could have added constraints:
        \begin{itemize}
            \item Only 2 projects can be selected ($X_1+X_2+X_3+X_4 = 2$).
            \item If project 1 is selected, then project 3 must also be selected ($X_1 \leq X_3$)
            \item If project 2 is selected, then project 4 can't be selected ($X_2+X_4 \leq 1$) - He acknowledged the fact that this condition goes the other way around as well.
        \end{itemize}
\end{itemize}

\subsection*{Example 3 - Set covering problem}
\begin{figure}[ht]
    \centering
    \includegraphics*[width=0.69\textwidth]{example3}
    \caption{Set covering problem}
    \label{fig: Set covering problem}
\end{figure}
Find min number of hospitals that can be built. Trying to give each city access to a hospital(They have to be adjacent to simulate access).
\begin{equation}
   Let X_j =
  \begin{cases}
    1 & \text{if hospital built in $j; j$ = 1,2,3..., 11} \\
    0 & \text{Otherwise}
  \end{cases}
\end{equation}

Min $X_1+X_2+...+X_{11}$\\
St:\\
$X_1+X_2+X_3+X_4 \geq 1 (X1)\\
X_1+X_2+X_3+X_5 \geq 1 (X2)\\
X_1+X_2+X_3+X_5+X_6 \geq 1 (X3)\\
X_1+X_3+X_6+X_7 \geq 1 (X4)\\
X_2+X_3+X_6+X_8+X_9 \geq 1 (X5)\\
X_3+X_4+X_5+X_7+X_8+X_9 \geq 1 (X6)\\
X_4+X_6+X_8 \geq 1 (X7)\\
X_5+X_6+X_7+X_9+X_{10} \geq 1 (X8)\\
X_5+X_6+X_8+X_{10}+X_{11} \geq 1 (X9)\\
X_8+X_9+X_{11} \geq 1 (X10)\\
X_9+X_{10} \geq 1 (X11)\\
X_1, X_2, X_3, ... X_{11} \in \{0,1\} 
$
\begin{itemize}
    \item To get the constraints, we go from $X_1 to X_{11}$ - For each X, the sum of it and all adjacent (touching) X's must at least 1. 
    \item This means we build at least one hospital to accomodate that group of cities. This will eventually cover all combinations of X's thus we'll be sure we ran through every possible hospital location (aka covered all cities).
    \item Optimal solution - $X_3=X_8=X_9=1$
\end{itemize}

\subsection*{Example 4 - Fixed charge problem}
\begin{figure}[ht]
    \centering
    \includegraphics*[width=0.65\textwidth]{example4}
    \caption{Fixed charge problem}
    \label{fig: Fixed charge problem}
\end{figure}
Let $X_j$ be the  number of product j produced.
\begin{equation}
   Let Y_j =
  \begin{cases}
    1 & \text{if product is produced; $j$ = 1,2,3} \\
    0 & \text{Otherwise}
  \end{cases}
\end{equation}
Max 48$X_1+55X_2+50X_3-1000Y_1-800Y_2-900Y_3$\\
St:\\
$
2X_1+3X_2+6X_3 \leq 600\\
6X_1+3X_2+4X_3 \leq 300\\
5X_1+6X_2+2X_3 \leq 400\\
X_1 \leq M_1Y_1\\
X_2 \leq M_2Y_2\\
X_3 \leq M_3Y_3\\
Y_1,Y_2,Y_3 \in \{0;1\}\\
X_1,X_2,X_3 \geq 0 $ and Integer

\begin{itemize}
    \item Manufacturing where you're making paint
    \item \textbf{Setup cost (fixed charge)} - Cost of cleaning machine after making red paint tp prepare to make yellow for example.
    \item \textbf{Variable cost} - e.g Hiring trucks will cost a certain, fixed, amount (fixed cost) and the cost of using them will vary (hence variable cost).
    \item The 4th to 6th constraints are added to make sure that each cost $Y_j$ is only applied if $X_j$ is actually produced.
        \begin{itemize}
            \item M is a big number greater than 0.
            \item if X is 0, Y can be 0 or 1.
            \item BUT if X is 1, Y \textbf{HAS} to be 1 as well (Since MY has to be at least equal to X).
            \item This is to say, surely if we produce at least one (X), there has to be some cost involved (Y).
            \item In fact, the moment X is slightly above 0, Y is 1.
            \item M is acting as an upper bound for X.
            \item In this case, We can find a reasonable value for M1:
                \begin{itemize}
                    \item Method - For each X, in the first 3 constraints, if the other X values are 0, M will be the max value X can take:
                        \begin{itemize}
                            \item For X1,
                            \item For constraint 1: X1 = 300
                            \item For constraint 2: X1 = 50
                            \item For constraint 3: X1 = 80
                            \item Therefore, the max it can be will be the min of the 3 values (\textbf{We do this so that all constraints are satisfied}; Therefore, M1=50).
                        \end{itemize}
                \end{itemize}
        \end{itemize}
\end{itemize}

\subsection*{Relaxation of ILP}
\begin{figure}[ht]
    \centering
    \includegraphics*[width=0.65\textwidth]{Relaxation_of_ILP}
    \caption{Relaxation of ILP}
    \label{fig: Relaxation of ILP}
\end{figure}
\begin{itemize}
    \item When you drop the integer constraints, you relax the integer constraint, hence LP relaxation.
    \item If the solution we find just happens to be an integer solution, we're done, else, we have to do something else.
    \item The feasible region of the relaxed problem is a \textbf{superset} (contains) of the associated integer linear programming solution feasible region. This makes sense as restricting things can only make the feasible region smaller.
    \item The objective func value for the integer solution can't be greater than that of the relaxed (normal simplex) version in a maximization.
    \item The optimal solution of the relaxed solution is the upper bound of the ILP solution for a maximization.
    \item The optimal solution of the relaxed solution is the lower bound of the ILP solution for a minimization.
    \item At each iteration in our search for a solution, the objective function value for any feasible solution ILP is a lower bound for an maximixation (We can't accept anything lower since we already have at least that value as a potential solution).
    \item At each iteration in our search for a solution, the objective function value for any feasible solution ILP is an upper bound for an minimization (We can't accept anything above since we already have at most that value as a potential solution).
    \item This is the \textbf{Branch and Bound} algorithm.
    \item It involves solving a series of LP problems.
    \item It can solve any ILP; In fact, any optimization (It's just that some take forever).
    \item In practice, the branch and bound can take a lot of computational effort and time.
\end{itemize}
\pagebreak

\subsection*{Example using the Branch and Bound algorithm}
\begin{figure}[ht]
    \centering
    \includegraphics*[width=0.69\textwidth]{Branch_and_Bound_example1}
    \caption{Branch and Bound example 1}
    \label{fig: Branch and Bound example 1}
\end{figure}
\pagebreak

\begin{figure}[ht]
    \centering
    \includegraphics*[width=1.2\textwidth]{Branch_and_Bound_workedexample_part1}
    \caption{Branch and Bound worked example part 1}
    \label{fig: Branch and Bound worked example part 1}
\end{figure}
\pagebreak

\begin{figure}[ht]
    \centering
    \includegraphics*[width=1.2\textwidth]{Branch_and_Bound_workedexample_part2}
    \caption{Branch and Bound worked example part 2}
    \label{fig: Branch and Bound worked example part 2}
\end{figure}
\begin{itemize}
    \item If both decision variables are not integers, branch arbitrarilly on any one of them. You will get to the same solution eventually.
\end{itemize}
\pagebreak

\end{document}